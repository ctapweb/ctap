\documentclass{beamer}
\usepackage[utf8]{inputenc}
\usepackage{graphicx}
\usepackage{natbib}
\usepackage{url}
\usetheme{Madrid}
\usecolortheme{beaver}
\bibliographystyle{apalike}

\title[CTAP Demo]{CTAP: A Web-Based Tool Supporting Automatic Complexity
Analysis}
\author[X.B. Chen \& D. Meurers]{Xiaobin Chen, \\
	xiaobin.chen@uni-tuebingen.de \\
	\and 
	Detmar Meurers, \\
	dm@sfs.uni-tuebingen.de
}
\institute[T\"ubingen]{T\"ubingen University}
\date{\today}

\begin{document}
	 \frame{\titlepage}

	\begin{frame}
		\frametitle{Linguistic analysis of texts}

		(Automatic) Linguistic analysis has been widely used for: 
		\begin{itemize}
			\item assessing text readability 
			\item modeling processing difficulty of sentences
			\item analyzing/scoring student writings
			\item comparing languages typologies and their historical development
			\item attributing authorship
			\item identifying native languages
			\item detecting plagiarism
			\item assessing answers to questions
			\item predicting diseases
			\item ...
		\end{itemize}
	\end{frame}


	\begin{frame}
		\frametitle{Existing tools for for text analysis}

		A number of tools have been released in the past few years. e.g.

		\begin{itemize}
			\item Syntactic and Lexical Complexity Analyzers \citep{Lu-10}
			\item Cohmetrix \citep{McNamara.ea-14}
			\item Suite of Linguistic Analysis Tools
			\citep{Crossley.ea-16a,Crossley.ea-16b}, also
			\url{http://www.kristopherkyle.com/tools.html}
			\item Computerized Propositional Idea Density Rater
			\citep[CPIDR]{Brown.ea-08}. 
			\item ETS's TextEvaluator
			\url{https://texteval-pilot.ets.org/TextEvaluator/}
			\item Pearson's Reading Maturity Metric
			\item Text Analysis, Crawling, and interpretation Tool
			\citep[TACIT]{Dehghani.ea-16}
		\end{itemize}
	\end{frame}

	\begin{frame}
		\frametitle{Problems with existing tools}

		\begin{itemize}
			\item Duplicated work: significant amount of feature overlap
			\item (Re)Usability of tools and analysis components
				  \begin{itemize}
					  \item OS-dependent standalone deploy
					  \item Source code release hard to use for non-programmers
					  \item Unfriendly user interface
				  \end{itemize}
			\item Feature proliferation, e.g.
				  \begin{itemize}
					  \item CohMetrix: 106 metrics
					  \item \citet{Housen-15}: $>$200 features for measuring L2
					  complexity
					  \item \citet{Vajjala-15}: $>$200 features for readability
					  assessment
				  \end{itemize}
		\end{itemize}
	\end{frame}

	\begin{frame}
		\frametitle{System demands}

		A system that is: 

		\begin{itemize}
			\item Web-based
			\item user-friendly
			\item modularized and reusable in terms of analysis components
			\item supports real-life use by ordinary users
		\end{itemize}
	\end{frame}

	\begin{frame}
		\frametitle{CTAP System Architecture}

		\centering
		\includegraphics[width=.8\textwidth]{img/ctap_architecture}
	\end{frame}

	\begin{frame}
		\frametitle{Corpus Manager}

		\begin{columns}
			\column{0.4\textwidth}
				 \centering
				  \includegraphics[height=.7\textheight]{img/corpus_manager}

			\column{0.6\textwidth}
				  Helps users manage the language materials that need to be
				  analyzed. 
				  \begin{itemize}
					  \item use folders to groups corpora
					  \item use corpora to hold texts
					  \item use tags to label texts based on e.g. document genre,
					  target reader levels, etc.
				  \end{itemize}

		\end{columns}
	\end{frame}

	\begin{frame}
		\frametitle{Feature Selector}

		\begin{columns}
			\column{0.4\textwidth}
				  \includegraphics[height=.7\textheight]{img/feature_selector}

			\column{0.6\textwidth}
				The Feature Selector supports: 
				  \begin{itemize}
					  \item creating feature set to hold selected features
					  \item add/remove features from feature set
				  \end{itemize}

				  Developers are encouraged to participate in in feature
				  development at \url{https://github.com/ctapweb}.
		\end{columns}
	\end{frame}

	\begin{frame}
		\frametitle{Analysis Generator}

		\begin{columns}
			\column{0.4\textwidth}
				  \includegraphics[height=.7\textheight]{img/analysis_generator}

			\column{0.6\textwidth}
				 Each analysis extracts a set of features from the designated
				 corpus. 
				 The analysis generator is used to:
				  \begin{itemize}
					  \item create new analyses
					  \item run analyses and monitor their progress
					  \item export analysis results in CSV format
				  \end{itemize}
		\end{columns}
	\end{frame}


	\begin{frame}
		\frametitle{Result Visualizer}

				  \includegraphics[width=\textwidth]{img/result_visualizer}

					The Result Visualizer is a simple and intuitive module that
					plots analysis results for the user to visualize preliminary
					findings from the analysis.
	\end{frame}
	
	\begin{frame}
		\frametitle{Design features of CTAP}
		
		\begin{itemize}
			\item Consistent, easy-to-use, friendly user interface
			\item Modularized, reusable, and collaborative development of analysis
			components
			\item Flexible corpus and feature management
		\end{itemize}

	\end{frame}

	\begin{frame}
		\frametitle{System demo}

		\centering
		\includegraphics[width=.8\textwidth]{img/ctapweb}

		\url{http://ctapweb.com}
	\end{frame}

	\begin{frame}
		\frametitle{Outlook}

		\begin{itemize}
			\item Populating the system with more features
			\item Replicating studies that involved text analysis to validate the
			system and identify other function needs 
			\item Model construction functionality (machine learning)
		\end{itemize}

		More details available in the paper:

		\scriptsize{
		Chen, X.B., Meurers, D. (2016). CTAP: A Web-Based Tool Supporting
		Automatic Complexity Analysis. In \textit{Proceedings of The Workshop on
		Computational Linguistics for Linguistic Complexity}. Osaka, Japan. The
		International Committee on Computational Linguistics.}

	\end{frame}

	\begin{frame}
		\frametitle{References}
		\tiny
		\bibliography{bibliography}

	\end{frame}
	
\end{document}

